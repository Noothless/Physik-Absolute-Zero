\documentclass[12pt,a4paper]{article}

\usepackage{a4wide}
\usepackage{fancyhdr}
\usepackage{graphicx}
\usepackage{epsfig}
\usepackage{parskip}
\usepackage[ansinew]{inputenc}
\usepackage{amsmath}
\usepackage{amssymb}
\usepackage{bm}
\usepackage[free-standing-units=true]{siunitx} % for consistent handling of SI units
\usepackage[colorlinks=true, pdfstartview=FitV, linkcolor=blue, citecolor=blue, urlcolor=blue]{hyperref} % enable links

\setlength{\parindent}{0pt}

\newcommand{\m}[1]
{\mathrm{#1}}

\title{Title of the experiment}
\author{Authors}
\date{\today }

\begin{document}

\maketitle

\begin{abstract}
The abstract is a short summary of the experiment. It should
motivate the reader to actually read the whole paper. The length of
the abstract should not exceed 10 lines. Every section of the report
should be represented by one or two sentences. This includes the
results and the conclusions. There is no need to keep the reader in
suspense over the findings that you want to present. Often the
abstract is written last.
\end{abstract}


\tableofcontents

\section{Introduction}

In general this section should tell the reader why he or she should
be interested in your paper. Give some background to the
experiment, and describe the underlying principles. This is typically where you provide references to previous publications~\cite{Sato2003,Cross2004}.

\section{Experiment}

This section is where you describe what you actually did in the lab.
You explain what data you measured and how. Here, you might want to
provide a sketch of the experimental setup, if this is applicable.
This sketch could be taken from the experiment manual. In particular,
you want to define all parameters that you use or measure in this
experiment.

In general the information you provide in this section should be
sufficient for the reader to reproduce your results.

\section{Results}

This paragraph is where you present the results from your
experiment. This could be in the form of a table (if only very few
parameters where measured) or as a figure. In the text you should
essentially describe what can be seen in the figures, i.e. explain
your axis and how the dependent variable changes as a function of
the independent variable. Discuss trends of the data as well as the
magnitude, origin and nature of the experimental uncertainties. Keep
in mind that the measurement results are always correct! They might
just not be the answer to the question you had in mind.

Depending on the experiment the discussion of the results can
also be a part of the data analysis section.

\section{Data analysis}

\begin{figure}
\centering
\includegraphics[width=\textwidth]{test.pdf}
\caption{In the figure caption you describe what is plotted in the
figure. In this case: The measured data is depicted by the blue dots
with error bars indicating the statistical error. The red line is a
linear fit to the data.} \label{fig1}
\end{figure}

In the data analysis section you describe the post processing of the
data. How did you obtain the data that you plot in the figures? The raw data does not necessarily need
to be presented in the report. An important part of this paragraph
are the measurement uncertainties. You should provide the
uncertainties of all experimental results, i.e. in the form of error
bars. Further, you should explain the origin of these uncertainties.

All figures or tables that are part of your report have to be
referenced somewhere in the text, ideally in order of their
appearance (``as shown in Fig.~\ref{fig1}''). Figures have to
have axis labels with units and a sensible scale. If more than on
data set is plotted you need to provide a legend. This may be a sentence in the caption (``red dots denote data measured with \SI{1}{\milli\volt}, blue crosses were measured with \SI{10}{\milli\volt}''). Make sure to use a consistent style for all referencing and citing. The best method is to use reference commands like those shown here, which also provide clickable links.

After you have presented the data you need to interpret it. To this
end you want to discuss the theoretical model that describes your
data and you will derive model parameters from your measurement data
(i.e. by fitting it to the data). There are many way how you can include equations and mathematical terms in your report. The easiest is to write them inside the text like this: $\Gamma =\SI{1.5}{\micro\meter\per\square\second}$. If you need to write a long equation, it is recommended to use e.g. the align environment.
\begin{align}\label{eq:Gamma}
    \Gamma = \frac{a}{4\kappa}\times ...
\end{align}
Such an equation can be referenced as Eq.~\eqref{eq:Gamma}. Here, you will again elaborate on
the confidence interval of the derived values (error propagation).
This is an important part of the report and it will be the basis for
the next paragraph.

\section{Discussion}

So far you have discussed how you have obtained your data and the
quantities you derived from it. In this section you should discuss
the results in the context of physical laws. Depending on the
experiment you want to compare your result and its uncertainty with
the literature value. If you want to confirm a physical model that
explains a certain phenomenon you want to assess if this model
describes the data well within the confidence intervals, or whether
a simpler model describes the data just as well.

\section{Conclusion}

In the concluding paragraph you summarize the result, with the
emphasis on what you have discovered in this work. You can end this
with an outlook on future research, i.e. how could the results be
improved or what would be a logical follow up experiment.

\newpage

\section{Dos and Don'ts}

\begin{itemize}

\item Be honest with yourself and with the reader. Try to find possible loopholes in your conclusion and explicitly mention them.

\item Be aware that \textbf{scientific fraud} is an important topic that we (and the entire ETH) take very seriously. There are many forms of fraud, from copying text without referencing it to forging data. If unsure, ask your assistants about specific issues.

\item Good writing is largely a question of practice and of experience. Why not read some scientific papers to study how professionals write? We are happy to recommend some literature. 

\item A good practice is to begin each paragraph with a `topic sentence' that conveys the main message of the paragraph. As an example, the experimental section might start with ``We performed experiments with a mechanical resonator inside a vacuum chamber.'' From this short sentence, the reader gathers immediately that the paragraph is about the experimental setup.

\item Avoid using passive voice for extended paragraphs. The use of active language can make the text more interesting to read and is by default preferred by many English writers. For instance, instead of writing ``The data points were measured over the course of 405 seconds'', you can write ``We measured data points over the course of 405 seconds'' or simply ``The data acquisition lasted 405 seconds''. Of course, sometimes it may be better to use passive voice in order to describe basic processes, e.g. ``Samples were cleaned for 3 minutes in acetone''. The choice is yours - try out what fits better in specific cases.

\item Write in short sentences. Always put clarity before artistic form. Whenever possible, avoid interrupting your sentences with brackets, formulas, or complex mathematical signs. Your text is much easier to read when you group such additional information at the end of a sentence, in a table, or in the caption of a figure. Remember that your readers might need their full attention for the physics involved (and do not want to decipher complex sentences).

\item Avoid slang and terms that might not be known to the reader. One of the most difficult tasks is to explain something very complicated in simple terms that newspaper readers might understand. If you have to use specialized terms, try to explain them when they first appear.

\item When you use abbreviations like `AFM', make sure you use the full term once. ``Nanoscale surfaces can be characterized with an atomic force microscope (AFM).''

\item As a rule, use ``cannot'' instead of ``can't'',  ``will not'' instead of ``won't'', ``do not'' instead of ``don't'' and so on (the title of this section is an exception).

\item Graphs should not be overloaded with information. Make the essential features stand out. Presenting scientific data is an art!

\item Graphs should be drawn with the help of a software. In any case, graphs have to fulfill all relevant criteria of good scientific practice, such as well-scaled and labeled axes (including units), and the data points must be clearly
visible and contain error bars where applicable.

\item Fitting parameters only need to be provided for actual physical models, not for a ``guide to the eye''. Measured and derived values should be given with error bars (confidence intervals) and an appropriate number of significant digits.

\item Figure captions are an important part of a figure. Ideally, a reader that is familiar with the field should understand your results by merely looking at your figures and reading the captions. Figure captions are an ideal place to give specific numbers that are not absolutely required in the main text (such as `applied voltage' or `laser power').

\item The reports should be written with a text processing software (e.g. Latex).

\end{itemize}


\begin{thebibliography}{99}

% =======================================================================================
% =======================================================================================

\bibitem{Sato2003}
M. Sato, B. E. Hubbard, A. J. Sievers, B. Ilic, D. A. Czaplewski, and H.G. Craighead, Phys. Rev. Lett. \textbf{90}, 044102 (2003).

% =======================================================================================

\bibitem{Cross2004}
M. C. Cross, A. Zumdieck, R. Lifshitz, and J. L. Rogers, Phys. Rev. Lett. \textbf{93}, 224101 (2004).


\end{thebibliography}


\end{document}