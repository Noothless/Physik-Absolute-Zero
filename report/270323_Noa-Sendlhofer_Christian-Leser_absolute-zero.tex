\documentclass[12pt,a4paper]{article}

\usepackage[margin=1in]{geometry}
\usepackage{fancyhdr}
\usepackage{graphicx}
\usepackage{epsfig}
\usepackage{parskip}
\usepackage[ansinew]{inputenc}
\usepackage{amsmath}
\usepackage{amssymb}
\usepackage{bm}
\usepackage{float}
\usepackage[free-standing-units=true]{siunitx} % for consistent handling of SI units
\usepackage[colorlinks=true, pdfstartview=FitV, linkcolor=blue, citecolor=blue, urlcolor=blue]{hyperref} % enable links

\setlength{\parindent}{0pt}

\newcommand{\m}[1]
{\mathrm{#1}}

\title{Experiment 9: absolute zero temperature}
\author{Noa Sendlhofer, Christian Leser}
\date{\today}

\begin{document}

\maketitle

\begin{abstract}
    
    % The abstract is a short summary of the experiment. It should
    % motivate the reader to actually read the whole paper. The length of
    % the abstract should not exceed 10 lines. Every section of the report
    % should be represented by one or two sentences. This includes the
    % results and the conclusions. There is no need to keep the reader in
    % suspense over the findings that you want to present. Often the
    % abstract is written last.
\end{abstract}

\tableofcontents

\section{Introduction}
    Starting from the Boyle-Mariotte law $pV = \text{const}$, stated in 1661 by R. Townley, the question what happens if the pressure is changed while keeping the same volume is raised.
    G. Amontons delivered an answer to this question in 1703: pressure and temperature change are linked linearly.
    \begin{align}
        p(t) = \text{const} (t - t_0)
    \end{align}
    As pressures cannot be negative, it is now obvious that there must be a 'lowest' temperature, hereby referred to as the absolute zero point of temperature. 
    The findings also imply that pressure depends on temperature, amount of particles and volume, leading to the ideal gas law~\ref{eq_igl}.
    To determine the absolute zero point of temperature, we make use of the findings from Townley and Amontons:
    An apparatus of constant volume to which a pressure measurement device is connected can be heated up and cooled down, resulting in a change of pressure in the closed chamber.
    From these changes in pressure, a linear relation can be established which leads to the value of the lowest possible temperature $t_0$

    % In general this section should tell the reader why he or she should
    % be interested in your paper. Give some background to the
    % experiment, and describe the underlying principles. This is typically where you provide references to previous publications~\cite{Sato2003}.

\section{Experiment}
    % This section is where you describe what you actually did in the lab.
    % You explain what data you measured and how. Here, you might want to
    % provide a sketch of the experimental setup, if this is applicable.
    % This sketch could be taken from the experiment manual. In particular,
    % you want to define all parameters that you use or measure in this
    % experiment.

    % In general the information you provide in this section should be
    % sufficient for the reader to reproduce your results.

    To determine the absolute zero point of the temperature, the apparatus in \textbf{Insert figure} was used.
    


\section{Methods}

    % This paragraph is where you present the results from your
    % experiment. This could be in the form of a table (if only very few
    % parameters where measured) or as a figure. In the text you should
    % essentially describe what can be seen in the figures, i.e. explain
    % your axis and how the dependent variable changes as a function of
    % the independent variable. Discuss trends of the data as well as the
    % magnitude, origin and nature of the experimental uncertainties. Keep
    % in mind that the measurement results are always correct! They might
    % just not be the answer to the question you had in mind.

    % Depending on the experiment the discussion of the results can
    % also be a part of the data analysis section.


\section{Data Analysis}

    % In the data analysis section you describe the post processing of the
    % data. How did you obtain the data that you plot in the figures? The raw data does not necessarily need
    % to be presented in the report. An important part of this paragraph
    % are the measurement uncertainties. You should provide the
    % uncertainties of all experimental results, i.e. in the form of error
    % bars. Further, you should explain the origin of these uncertainties.

    % All figures or tables that are part of your report have to be
    % referenced somewhere in the text, ideally in order of their
    % appearance (``as shown in Fig.~\ref{fig1}''). Figures have to
    % have axis labels with units and a sensible scale. If more than on
    % data set is plotted you need to provide a legend. This may be a sentence in the caption (``red dots denote data measured with \SI{1}{\milli\volt}, blue crosses were measured with \SI{10}{\milli\volt}''). Make sure to use a consistent style for all referencing and citing. The best method is to use reference commands like those shown here, which also provide clickable links.

    % After you have presented the data you need to interpret it. To this
    % end you want to discuss the theoretical model that describes your
    % data and you will derive model parameters from your measurement data
    % (i.e. by fitting it to the data). There are many way how you can include equations and mathematical terms in your report. The easiest is to write them inside the text like this: $\Gamma =\SI{1.5}{\micro\meter\per\square\second}$. If you need to write a long equation, it is recommended to use e.g. the align environment.
    %     \begin{align}\label{eq:Gamma}
    %         \Gamma = \frac{a}{4\kappa}\times ...
    %     \end{align}
    % Such an equation can be referenced as Eq.~\eqref{eq:Gamma}. Here, you will again elaborate on
    % the confidence interval of the derived values (error propagation).
    % This is an important part of the report and it will be the basis for
    % the next paragraph.

\section{Discussion}
    We have determined the absolute zero temperature to be \textbf{insert value}.
    The literature value of $t_{0, \text{lit}} = -273.15 \si{\celsius}$, found on the website britannica \cite{literature_absolute_zero} is within our error margins.
    To depict one major use of the calculated value, it is possible to determine a scientific temperature scale known as the Kelvin scale.
    It takes the absolute zero temperature as the starting point while maintaining the same step between every value as the Celsius scale, so $0 \si{\kelvin} = -273.15 \si{\celsius}$.
    This leads to $1K = \frac{\Delta_{0, E}(T)}{273.16000...}$ with $\Delta_{0, E}(T)$ being the difference in temperature between the absolute zero and the freezing point of water.
    The derived scale is used in domains like thermodynamics.

    We made use of our value for $t_0$ by determining the temperature of liquid nitrogen and obtained a value of \textbf{insert value}.
    Again, our margins cover the literature value of $t_{LN2, \text{lit}} = -198.79 \si{\celsius}$ found on britannica \cite{literature_liquid_nitrogen}.
    Knowing the temperature of liquid nitrogen can be useful when in need of a coolant for future research.

    When comparing each obtained value with the literature values, one can be satisfied with the results the experimental setup delivers.

    Still, the error margin could be reduced.
    The biggest impact is given by the error of the slope from the linear relation, resulting from uncertainties in pressure and temperature.
    One portion of the uncertainty in pressure is due to the sensor.
    A higher precision sensor would reduce the error of the slope $\Delta C$.
    Moreover, the temperature change could affect the reading of the sensor, as it is close to the bulb.
    On the other hand, the further away the sensor is located from the helium bulb, the longer the glass tube connecting both.
    Within this tube, a portion of the Helium receives a different heat than the bulb leading to unevenly heated gas in the apparatus.
    That means the pressure in our system is not exactly proportional to the applied temperature.
    Also, the steam used to heat the bulb to the boiling point of water cools down on its way from the bottom to the top of the bulb, resulting in an uneven change of volume of the glassware.
    Therefore, the volume the gas takes up does not exactly match the calculated value.

    % So far you have discussed how you have obtained your data and the
    % quantities you derived from it. In this section you should discuss
    % the results in the context of physical laws. Depending on the
    % experiment you want to compare your result and its uncertainty with
    % the literature value. If you want to confirm a physical model that
    % explains a certain phenomenon you want to assess if this model
    % describes the data well within the confidence intervals, or whether
    % a simpler model describes the data just as well.


    
    
   

\section{Conclusion}
    In this experiment series, we determined the absolute zero temperature $t_0 = (-275.9 \pm 2.45)\si{\bf\celsius}$.
    After having calibrated the apparatus, it could be used as a thermometer and the temperature of liquid nitrogen $t_{LN2} = (-198.5 \pm 2.56)\si{\bf\celsius}$ was found.
    The obtained values are close to the literature values, which confirms the experimental setup to be an effective tool to determine $t_0$ and to use as a thermometer.
    It is possible to derive the Kelvin scale, which serves in different scientific domains, from the values measured and calculated in this experiment.
    However, there are ways to reduce the error margin:
    Waiting for the gas in the apparatus as well as the glass bulb to heat up or cool down is essential to getting the correct result.
    At the same time, the longer we wait, the more air could flow through tiny leaks in the connectons of the tubes.
    Better sealing of the apparatus would ensure less air flow, enable more time for the equilibrium to settle and in turn deliver a better precision of the results.
    Furthermore, monitoring the temperature and pressure of the lab close to the glass bulb while taking the measurement would result in higher certainty.
    Also, a material changing its form less than glass would bring the linear plot closer to the real values

    % In the concluding paragraph you summarize the result, with the
    % emphasis on what you have discovered in this work. You can end this
    % with an outlook on future research, i.e. how could the results be
    % improved or what would be a logical follow up experiment.

\section{Appendix}
    


\begin{thebibliography}{99}

\bibitem{STH}
Stuart Hunt \& Associates Ltd., Radioactive Material Safety Data Sheet Strontium-90, \href{https://www.stuarthunt.com/de/cache/modules_elements/103/Strontium-90-Sealed.pdf}{Website}

\bibitem{Sato2003}
M. Sato, B. E. Hubbard, A. J. Sievers, B. Ilic, D. A. Czaplewski, and H.G. Craighead, Phys. Rev. Lett. \textbf{90}, 044102 (2003).

\bibitem{Cross2004}
M. C. Cross, A. Zumdieck, R. Lifshitz, and J. L. Rogers, Phys. Rev. Lett. \textbf{93}, 224101 (2004).

\bibitem{Manual}
Manual to experiment 41 beta decay (2022).


\end{thebibliography}

\end{document}