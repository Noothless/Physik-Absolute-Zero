\section{Discussion}
    We have determined the absolute zero temperature to be \textbf{insert value}.
    The literature value of $t_{0, \text{lit}} = -273.15 \si{\celsius}$, found on the website britannica~\cite{literature_absolute_zero} is within our error margins.
    To depict one major use of the calculated value, it is possible to determine a scientific temperature scale known as the Kelvin scale.
    It takes the absolute zero temperature as the starting point while maintaining the same step between every value as the Celsius scale, so $0 \si{\kelvin} = -273.15 \si{\celsius}$.
    This leads to $1K = \frac{\Delta_{0, E}(T)}{273.16000\cdots}$ with $\Delta_{0, E}(T)$ being the difference in temperature between the absolute zero and the freezing point of water.
    The derived scale is used in domains like thermodynamics.

    We made use of our value for $t_0$ by determining the temperature of liquid nitrogen and obtained a value of \textbf{insert value}.
    Again, our margins cover the literature value of $t_{LN2, \text{lit}} = -198.79 \si{\celsius}$ found on britannica~\cite{literature_liquid_nitrogen}.
    Knowing the temperature of liquid nitrogen can be useful when in need of a coolant for future research.

    When comparing each obtained value with the literature values, one can be satisfied with the results the experimental setup delivers.

    Still, the error margin could be reduced.
    The biggest impact is given by the error of the slope from the linear relation, resulting from uncertainties in pressure and temperature.
    One portion of the uncertainty in pressure is due to the sensor.
    A higher precision sensor would reduce the error of the slope $\Delta C$.
    Moreover, the temperature change could affect the reading of the sensor, as it is close to the bulb.
    On the other hand, the further away the sensor is located from the helium bulb, the longer the glass tube connecting both.
    Within this tube, a portion of the Helium receives a different heat than the bulb leading to unevenly heated gas in the apparatus.
    That means the pressure in our system is not exactly proportional to the applied temperature.
    Also, the steam used to heat the bulb to the boiling point of water cools down on its way from the bottom to the top of the bulb, resulting in an uneven change of volume of the glassware.
    Therefore, the volume the gas takes up does not exactly match the calculated value.

    % So far you have discussed how you have obtained your data and the
    % quantities you derived from it. In this , you should discuss
    % the results in the context of physical laws. Depending on the
    % experiment you want to compare your result and its uncertainty with
    % the literature value. If you want to confirm a physical model that
    % explains a certain phenomenon you want to assess if this model
    % describes the data well within the confidence intervals, or whether
    % a simpler model describes the data just as well.


    
    
   