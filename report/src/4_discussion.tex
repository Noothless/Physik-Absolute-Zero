\section{Discussion}
    We have determined the absolute zero temperature to be \textbf{insert value}.
    The literature value of $t_{0, \text{lit}} = -273.15 \si{\celsius}$, found on the website britannica \cite{literature_absolute_zero} is within our error margins.
    To depict one major use of the calculated value, it is possible to determine a scientific temperature scale known as the Kelvin scale.
    It takes the absolute zero temperature as the starting point while maintaining the same step between every value as the Celsius scale, so $0 \si{\kelvin} = -273.15 \si{\celsius}$.
    This leads to $1K = \frac{\Delta_{0, E}(T)}{273.16000...}$ with $\Delta_{0, E}(T)$ being the difference in temperature between the absolute zero and the freezing point of water.
    The derived scale is used in domains like thermodynamics.

    We made use of our value for $t_0$ by determining the temperature of liquid nitrogen and obtained a value of \textbf{insert value}.
    Again, our margins cover the literature value of $t_{LN2, \text{lit}} = -198.79 \si{\celsius}$ found on britannica \cite{literature_liquid_nitrogen}.
    Knowing the temperature of liquid nitrogen can be useful when in need of a coolant for future research.

    When comparing each obtained value with the literature values, we are relatively satisfied with the result of our experiment

    % So far you have discussed how you have obtained your data and the
    % quantities you derived from it. In this section you should discuss
    % the results in the context of physical laws. Depending on the
    % experiment you want to compare your result and its uncertainty with
    % the literature value. If you want to confirm a physical model that
    % explains a certain phenomenon you want to assess if this model
    % describes the data well within the confidence intervals, or whether
    % a simpler model describes the data just as well.


    
    
   