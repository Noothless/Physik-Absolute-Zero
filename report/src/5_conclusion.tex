\section{Conclusion}
    In this experiment series, we determined the absolute zero temperature $t_0 = (-276 \pm 4)\si{\bf\celsius}$.
    After having calibrated the apparatus, it could be used as a thermometer and the temperature of liquid nitrogen $t_{LN2} = (-199 \pm 4)\si{\bf\celsius}$ was found.
    The obtained values are close to the literature values, which confirms the experimental setup to be an effective tool to determine $t_0$ and to use as a thermometer.
    It is possible to derive the Kelvin scale, which serves in different scientific domains, from the values measured and calculated in this experiment.
    However, there are ways to reduce the error margin:
    Waiting for the gas in the apparatus as well as the glass bulb to heat up or cool down is essential to getting the correct result.
    At the same time, the longer we wait, the more air could flow through tiny leaks in the connectons of the tubes.
    Better sealing of the apparatus would ensure less air flow, enable more time for the equilibrium to settle and in turn deliver a better precision of the results.
    Furthermore, monitoring the temperature and pressure of the lab close to the glass bulb while taking the measurement would result in higher certainty.
    Also, a material changing its form less than glass would bring the linear plot closer to the real values

    % In the concluding paragraph you summarize the result, with the
    % emphasis on what you have discovered in this work. You can end this
    % with an outlook on future research, i.e. how could the results be
    % improved or what would be a logical follow up experiment.