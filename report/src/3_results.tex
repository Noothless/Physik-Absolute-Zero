\section{Results}\label{sec_results}

    In the first step of this experiment, the pressure sensor has to be calibrated at known pressures and temperatures.
    The result of this calibration is presented in Fig.~\ref{fig_calibration}

    \begin{figure}[H]
        \centering
        \includegraphics[]{src/images/calibration.png}
        \caption{Calibration of the pressure sensor: A linear fit through $(U_L,p_L)$ and $(U_t,p_t)$ determines the slope $C$. $p_0$ shows the y-intersect at voltage $U_0 = 0$}
        \label{fig_calibration}
    \end{figure}

    $p_0$ can be found by entering a pair of known values into eq.~\ref{eq_pressure}, such as $(U_t,p_t)$. %p0 and C can be found by entering two pairs of known values into eq.. (such as ... ...) and then solving the linear system of equations.

    \begin{align}
        p_0 = p_t - C U_t \label{eq_p0}
    \end{align}

    Using the slope $C$ and y-offset $p_0$, which we determined in this first step, we can calculate the pressure
    of any sensor reading using formula \ref{eq_pressure}.

    \begin{align}
        p(U) = p_0 + CU \label{eq_pressure}
    \end{align}

    To determine the uncertainties in Eq.~\ref{eq_pressure}, we first have to calculate the errors in $p_0$, $C$ and $U$.
    Using the gaussian error propagation, we can then determine the error of the entire equation.


    From the manufacturer's datasheet, we know that the linearity of the sensor is guaranteed up to $\pm 0.05\%$ full scale. % of the full scale?
    This means that the sensor signal error is at most $0.05\%$ of $150\si{\milli\volt}$, thus $0.075\si{\milli\volt}$.

    Furthermore, it has to be taken into account, that the temperature of the sensor may vary from when
    it was calibrated and when the measurement was made. We can assume a voltage difference of $\pm 0.08\si{\milli\volt}/\si{\celsius}$ at $p = 0$.
    The slope of this curve can vary by $\pm 0.1\% \si{\celsius}$
    This is the uncertainty calculation for $\Delta U$:
    \begin{align}
        \Delta U &= 0.075 \si{\milli\volt} + 0.08 \frac{\si{\milli\volt}}{1 \si{\celsius} \pm 0.1 \% \si{\celsius}}\\
        &= 0.075 \si{\milli\volt} + 0.08 \si{\milli\volt} + 8 \cdot 10^{-5} \si{\milli\volt} \label{eq_dlta_u}\\
        \bf \Delta U &= \bf \pm 0.15508 ~\si{\bf\milli\volt} \label{dlta_u}
    \end{align}

    The errors in eq.~\ref{eq_dlta_u} were calculated using the gaussian error propagation. The full calculations of this error please refer
    to the jupyter notebook given here. %LINK GITHUB !!!!

    $\Delta C$ can be determined in a similar way:
    \begin{align}
        C &= \frac{p_L - p_t}{U_L - U_t}\\
        \Delta C &= \sqrt{ \left(\frac{\partial C}{\partial p_L} \Delta p_L \right)^2 +
                        \left(\frac{\partial C}{\partial p_t} \Delta p_t \right)^2 +
                        \left(\frac{\partial C}{\partial U_L} \Delta U_L \right)^2 +
                        \left(\frac{\partial C}{\partial U_t} \Delta U_t \right)^2 } \label{eq_dlta_c}
    \end{align}

    Using the errors for $U$ determined in eq. \ref{dlta_u}, as well as $\Delta p_t = \pm 0.1 \si{\milli\bar}$ given in the manual and $\Delta p_L = 200~\si{\pascal}$,
    we can determine a value for $\Delta C$ using eq.~\ref{eq_dlta_c}:
    \begin{align}
        \bf \Delta C = \pm 1.8779 \label{dlta_c}
    \end{align}

    Lastly, we have to determine the error for $\Delta p_0$. The error can be calculated in the same way as before, based on equation \ref{eq_p0}.
    Due to the fact that the calculated sensor error $\Delta U$ is full scale, it can be assumed for all sensor readings $U_i$.
    \begin{align}
        \Delta p_0 &= \sqrt{ \left(\frac{\partial p_0}{\partial C} \Delta C \right)^2 +
                            \left(\frac{\partial p_0}{\partial U_t} \Delta U_t \right)^2 +
                            \Delta p_t^2}\\
        \bf \Delta p_0 &= \bf \pm 131.407 ~\si{\bf\pascal} \label{dlta_p0}
    \end{align}

    We now have all the values for calculating $\Delta p(U)$. Based on eq. \ref{eq_pressure} we can determine the following error propagation: % On the base of eqref eq_pressure ?
    \begin{align}
        \Delta p(U) &= \sqrt{ \left(\frac{\partial p}{\partial C} \Delta C \right)^2 +
                            \left(\frac{\partial p}{\partial U} \Delta U \right)^2 +
                            \Delta p_0^2}\\
        &= \sqrt{ \left( U \Delta C \right)^2 +
                \left( C \Delta U \right)^2 + 
                \Delta p_0^2} \label{eq_dlta_p}
    \end{align}

    \begin{figure}[H]
        \centering
        \includegraphics[]{src/images/absolute_zero.png}
        \caption{Determination of absolute zero: Linear fit through $(t_E, p_E)$ and $(t_K, p_K)$ until $p = 0$ is reached.}
        \label{fig_abs_zero}
    \end{figure}
    Fig.~\ref{fig_abs_zero} is a visual representation of how the value for $t_0$ was obtained. Measuring the pressure of the Helium gas at two precisely
    defined temperatures will give us two points on a temperature/pressure graph. The y-intersection of the linear fit through these points will give us an approximative value for $t_0$.
    To get a more precise value, we have to take into account multiple factors like the expansion of the glass bulb under temperature changes, 
    as well as the influence of unevenly heated gas in the connecting tube between the bulb and the sensor.
    %Maybe a bit more context about why we show this figure, ex: as the properties are connected linearly, we get an approximative value for t_0. More factors have to be taken into account to determine the exact value of t_0.

    In this following chapter the results from the determination of absolute zero are presented. %in this following part? ist ja kein neues Kapitel
    Using eq.~\ref{eq_pressure} the resulting pressures were calculated from the raw sensor data.
    With the derived eq.~\ref{eq_dlta_p} we can then calculate the errors $\Delta p_K, \Delta p_E$. %With the equation derived in eqref...?
    \begin{align}
        p(U_E) &= 69630 ~\si{\pascal}\\
        p(U_K) &= 94535 ~\si{\pascal}\\
        \Delta p_E &= \pm 205.62 ~\si{\pascal} \label{val_pE}\\
        \Delta p_K &= \pm 248.23 ~\si{\pascal} \label{val_pK}
    \end{align}

    As described above the exact value for $t_0$ has to take into account the remaining gas
    in the tube connecting the bulb with the sensor, as well as the thermal expansion of the
    bulb itself. The following quadratic equation calculates the final value for $t_0$ while
    taking into account the factors stated above.
    \begin{align}
        a &= (1 + \varepsilon)p_E - (1 + \epsilon + \gamma t_K)p_K\\
        b &= \varepsilon(p_K - p_E)t_K + (1 + \gamma t_K)p_K t_L - p_E(t_L + t_K)\\
        c &= p_E t_L t_K\\
        t_0 &= \frac{-b \pm \sqrt{b^2 - 4a c}}{2a} \label{eq_t0}
    \end{align}

    Using a barometer in the lab we determined the boiling point of water to be $t_K = 98.323 \si{\celsius}$.
    The temperature in the lab was $t_L = 24 \si{\celsius}$.
    When entering the results from eq.~\ref{val_pE} and~\ref{val_pK}, as well as $t_K$ and $t_L$ into eq.~\ref{eq_t0},
    we get the following result for $t_0$:
    \begin{align}
        t_0 = -273.53~\si{\celsius} \label{val_t0}
    \end{align}

    The error calculation for $t_0$ was again accomplished using the gaussian error propagation.
    Due to the fact that $t_E$, as well as $t_K$, could be determined very precisely, we did not take
    the uncertainties of these values into account in the following error calculation.
    Therefore the only variables with errors are $p_E$, as well as $p_K$. % ... whose error margins were considered are...
    \begin{align}
        \Delta t_0 &= \sqrt{ \left(\frac{\partial t_0}{\partial p_E} \Delta p_E \right)^2 +
                            \left(\frac{\partial t_0}{\partial p_K} \Delta p_K \right)^2 }\\
        \Delta t_0 &= \pm 0.4427195~\si{\celsius} \label{dlta_t0}\\
        \bf t_0 &= \bf -273.53 \pm 0.44272~\si{\bf\celsius} \label{res_t0}
    \end{align}

    In the last part of the experiment, the apparatus was used as a thermometer, to determine the 
    temperature of liquid nitrogen.
    Using eq.~\ref{eq_pressure} the pressure $p_N$ was calculated from the raw sensor data $U_N$.
    \begin{align}
        p_N = 19635 \si{\pascal}
    \end{align}

    To calculate the temperature of liquid nitrogen we can use our known value of $t_0$ and solve
    the following equation for $t'_{LN2}$. This will only give us an approximate value, by simply solving a linear approach.
    \begin{align}
        \frac{p_E}{t_E - t_0} \approx& \; \frac{p_N}{t'_{LN2} - t_0}\\
        t'_{LN2} \approx& \; \frac{p_N}{p_E}(t_E - t_0) + t_0
    \end{align}

    Here we again have to take the volume of gas in the tube connecting the bulb with sensor,
    as well as the shrinkage of the glass bulb due to the temperature change into account.
    The following equation will give us an exact value for $t_{LN2}$:
    \begin{align}
        A \equiv& \; \frac{p_E}{t_E - t_0} + \frac{\varepsilon(p_E - p_N)}{t_L - t_0}\\
        =& \; 256.54~\si{\pascal/\celsius}\\
        t_{LN2} =& \; \frac{A t_0 + p_N}{A - \gamma p_N}\\
        =& \; -195.92~\si{\celsius} \label{val_tN}
    \end{align}

    Using the gaussian error propagation the uncertainty for $t_{LN2}$ was calculated. As stated above
    the uncertainties for the variables $t_L$ and $t_E$ are not considered in this calculation either.
    \begin{align}
        \Delta A =& \; \sqrt{ \left(\frac{\partial A}{\partial p_E} \Delta p_E \right)^2 +
                            \left(\frac{\partial A}{\partial p_N} \Delta p_N \right)^2 +
                            \left(\frac{\partial A}{\partial t_0} \Delta t_0 \right)^2 }\\
        =& \; 0.7715~\si{\pascal/\celsius}\\
        \Delta t_{LN2} =& \; \sqrt{ \left(\frac{\partial t_{LN2}}{\partial A} \Delta A \right)^2 +
                                    \left(\frac{\partial t_{LN2}}{\partial t_0} \Delta t_0 \right)^2 +
                                    \left(\frac{\partial t_{LN2}}{\partial p_N} \Delta p_N \right)^2 }\\
        =& \; 0.805704~\si{\celsius}\\
        \bf t_{LN2} =& \bf -195.92 \pm 0.80570 ~\si{\bf\celsius} \label{res_tN}
    \end{align}

    % Results
    % This paragraph is where you present the results from your
    % experiment. This could be in the form of a table (if only very few
    % parameters where measured) or as a figure. In the text you should
    % essentially describe what can be seen in the figures, i.e. explain
    % your axis and how the dependent variable changes as a function of
    % the independent variable. Discuss trends of the data as well as the
    % magnitude, origin and nature of the experimental uncertainties. Keep
    % in mind that the measurement results are always correct! They might
    % just not be the answer to the question you had in mind.

    % Depending on the experiment the discussion of the results can
    % also be a part of the data analysis section.



    % Data Analysis
    % In the data analysis section you describe the post processing of the
    % data. How did you obtain the data that you plot in the figures? The raw data does not necessarily need
    % to be presented in the report. An important part of this paragraph
    % are the measurement uncertainties. You should provide the
    % uncertainties of all experimental results, i.e. in the form of error
    % bars. Further, you should explain the origin of these uncertainties.

    % All figures or tables that are part of your report have to be
    % referenced somewhere in the text, ideally in order of their
    % appearance (``as shown in Fig.~\ref{fig1}''). Figures have to
    % have axis labels with units and a sensible scale. If more than on
    % data set is plotted you need to provide a legend. This may be a sentence in the caption (``red dots denote data measured with \SI{1}{\milli\volt}, blue crosses were measured with \SI{10}{\milli\volt}''). Make sure to use a consistent style for all referencing and citing. The best method is to use reference commands like those shown here, which also provide clickable links.

    % After you have presented the data you need to interpret it. To this
    % end you want to discuss the theoretical model that describes your
    % data and you will derive model parameters from your measurement data
    % (i.e. by fitting it to the data). There are many way how you can include equations and mathematical terms in your report. The easiest is to write them inside the text like this: $\Gamma =\SI{1.5}{\micro\meter\per\square\second}$. If you need to write a long equation, it is recommended to use e.g. the align environment.
    %     \begin{align}\label{eq:Gamma}
    %         \Gamma = \frac{a}{4\kappa}\times ...
    %     \end{align}
    % Such an equation can be referenced as Eq.~\eqref{eq:Gamma}. Here, you will again elaborate on
    % the confidence interval of the derived values (error propagation).
    % This is an important part of the report and it will be the basis for
    % the next paragraph.