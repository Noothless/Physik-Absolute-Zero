\section{Results}\label{sec_results}

    In the first step of this experiment, the pressure sensor has to be calibrated at known pressures and temperatures.
    The result of this calibration is presented in Fig.~\ref{fig_calibration}

    \begin{figure}[H]
        \centering
        \includegraphics[]{src/images/calibration.png}
        \caption{Calibration of the pressure sensor: A linear fit through $(U_L,p_l)$ and $(U_t,p_t)$ determines the slope $C$. $p_0$ shows the y-intersect at voltage $U_0 = 0$}
        \label{fig_calibration}
    \end{figure}

    $p_0$ can be found by entering a pair of known values into eq.~\ref{eq_pressure}, such as $(U_t,p_t)$.

    \begin{align}
        p_0 = p_t - C U_t \label{eq_p0}
    \end{align}

    Using the slope $C$ and y-offset $p_0$, we determined in this first step, we can calculate the pressure
    of any sensor reading using formula \ref{eq_pressure}.

    \begin{align}
        p(U) = p_0 + CU \label{eq_pressure}
    \end{align}

    To determine the uncertainties in Eq.~\ref{eq_pressure}, we first have to calculate the errors in $p_0$, $C$ and $U$.
    Using the gaussian error propagation, we can then determine the error of the entire equation.


    From the manufacturer's datasheet, we know that the linearity of the sensor is guaranteed up to $\pm 0.05\%$ full scale.
    This means that the sensor signal error is at most $0.05\%$ of $150\si{\milli\volt}$, thus $0.075\si{\milli\volt}$.

    Furthermore, it has to be taken into account, that the temperature of the sensor may vary from when
    it was calibrated and when the measurement was made. We can assume a voltage difference of $\pm 0.08\si{\milli\volt}/\si{\celsius}$ at $p = 0$.
    The slope of this curve can vary by $\pm 0.1\% \si{\celsius}$
    This is the uncertainty calculation for $\Delta U$:

    \begin{align}
        \Delta U &= 0.075 \si{\milli\volt} + 0.08 \frac{\si{\milli\volt}}{1 \si{\celsius} \pm 0.1 \% \si{\celsius}}\\
        &= 0.075 \si{\milli\volt} + 0.08 \si{\milli\volt} + 8 \cdot 10^{-5} \si{\milli\volt} \label{eq_dlta_u}\\
        \bf \Delta U &= \bf \pm 0.15508 ~\si{\bf\milli\volt} \label{dlta_u}
    \end{align}

    The errors in eq.~\ref{eq_dlta_u} were calculated using the gaussian error propagation.

    $\Delta C$ can be determined in a similar way:

    \begin{align}
        C &= \frac{p_L - p_t}{U_L - U_t}\\
        \Delta C &= \sqrt{ \left(\frac{\partial C}{\partial p_L} \Delta p_L \right)^2 +
                        \left(\frac{\partial C}{\partial p_t} \Delta p_t \right)^2 +
                        \left(\frac{\partial C}{\partial U_L} \Delta U_L \right)^2 +
                        \left(\frac{\partial C}{\partial U_t} \Delta U_t \right)^2 } \label{eq_dlta_c}
    \end{align}

    Using the errors for $U$ determined in eq. \ref{dlta_u}, as well as $\Delta p_t = \pm 0.1 \si{\milli\bar}$ given in the manual and $\Delta p_L = ???$,
    we can determine a value for $\Delta C$ using eq.~\ref{eq_dlta_c}:

    \begin{align}
        \bf \Delta C = 1.155 \label{dlta_c}
    \end{align}

    Lastly, we have to determine the error for $\Delta p_0$. The error can be calculated in the same way as before, using the equation \ref{eq_p0}.

    \begin{align}
        \Delta p_0 &= \sqrt{ \left(\frac{\partial p_0}{\partial C} \Delta C \right)^2 +
                            \left(\frac{\partial p_0}{\partial U_t} \Delta U_t \right)^2 +
                            \Delta p_t^2}\\
        \bf \Delta p_0 &= \bf 119.001 ~\si{\bf\pascal} \label{dlta_p0}
    \end{align}

    We now have all the values for calculating $\Delta p(U)$. Using eq. \ref{eq_pressure} we can determine the following error propagation:

    \begin{align}
        \Delta p(U) &= \sqrt{ \left(\frac{\partial p}{\partial C} \Delta C \right)^2 +
                            \left(\frac{\partial p}{\partial U} \Delta U \right)^2 +
                            \Delta p_0^2}\\
        &= \sqrt{ \left( U \Delta C \right)^2 +
                \left( C \Delta U \right)^2 + 
                \Delta p_0^2} \label{eq_dlta_p}
    \end{align}

    In this following chapter the results from the determination of absolute zero are presented.
    Using eq.~\ref{eq_pressure} the resulting pressures were calculated from the raw sensor data.

    \begin{align}
        p(U_E) = 
    \end{align}

    % Results
    % This paragraph is where you present the results from your
    % experiment. This could be in the form of a table (if only very few
    % parameters where measured) or as a figure. In the text you should
    % essentially describe what can be seen in the figures, i.e. explain
    % your axis and how the dependent variable changes as a function of
    % the independent variable. Discuss trends of the data as well as the
    % magnitude, origin and nature of the experimental uncertainties. Keep
    % in mind that the measurement results are always correct! They might
    % just not be the answer to the question you had in mind.

    % Depending on the experiment the discussion of the results can
    % also be a part of the data analysis section.



    % Data Analysis
    % In the data analysis section you describe the post processing of the
    % data. How did you obtain the data that you plot in the figures? The raw data does not necessarily need
    % to be presented in the report. An important part of this paragraph
    % are the measurement uncertainties. You should provide the
    % uncertainties of all experimental results, i.e. in the form of error
    % bars. Further, you should explain the origin of these uncertainties.

    % All figures or tables that are part of your report have to be
    % referenced somewhere in the text, ideally in order of their
    % appearance (``as shown in Fig.~\ref{fig1}''). Figures have to
    % have axis labels with units and a sensible scale. If more than on
    % data set is plotted you need to provide a legend. This may be a sentence in the caption (``red dots denote data measured with \SI{1}{\milli\volt}, blue crosses were measured with \SI{10}{\milli\volt}''). Make sure to use a consistent style for all referencing and citing. The best method is to use reference commands like those shown here, which also provide clickable links.

    % After you have presented the data you need to interpret it. To this
    % end you want to discuss the theoretical model that describes your
    % data and you will derive model parameters from your measurement data
    % (i.e. by fitting it to the data). There are many way how you can include equations and mathematical terms in your report. The easiest is to write them inside the text like this: $\Gamma =\SI{1.5}{\micro\meter\per\square\second}$. If you need to write a long equation, it is recommended to use e.g. the align environment.
    %     \begin{align}\label{eq:Gamma}
    %         \Gamma = \frac{a}{4\kappa}\times ...
    %     \end{align}
    % Such an equation can be referenced as Eq.~\eqref{eq:Gamma}. Here, you will again elaborate on
    % the confidence interval of the derived values (error propagation).
    % This is an important part of the report and it will be the basis for
    % the next paragraph.