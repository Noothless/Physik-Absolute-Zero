\section{Methods}
    % This section is where you describe what you actually did in the lab.
    % You explain what data you measured and how. Here, you might want to
    % provide a sketch of the experimental setup, if this is applicable.
    % This sketch could be taken from the experiment manual. In particular,
    % you want to define all parameters that you use or measure in this
    % experiment.

    % In general the information you provide in this section should be
    % sufficient for the reader to reproduce your results.

    We have multiple goals in this experiment series:
    \begin{enumerate}
        \item The calibration of the voltmeter and its corresponding pressures.
        \item Determination of the absolute zero point of temperature.
        \item Determination of the temperature of liquid nitrogen.
    \end{enumerate}
    We will achieve these results by making use of the linearly corresponding properties of ideal gases with the help of the apparatus shown in Fig.~\ref{fig_setup}.

    \begin{figure}[H]
        \centering
        \includegraphics[]{src/images/experimental_setup.png}
        \caption{experimental setup: further explain}
        \label{fig_setup}
    \end{figure}

    Starting with the calibration of the voltmeter, the voltages at room pressure and when a vacuum pump is connected are noted and compared to the actual pressure in the room and the pressure, the pump is able to maintain.
    Hereafter, the glass bulb is evacuated and filled with helium, the reason being that the properties of helium are closer to ideal gases than air.
    Moreover, the oxygen from the air would liquefy when cooled down to the temperature of liquid nitrogen (which would happen in the third step of the experiment series) and liquids do not act according to the ideal gas equation.

    In the next step, the helium-filled glass bulb is heated to the boiling point of water using steam while leaving the apparatus open so that excess helium can escape.
    We chose the temperature of boiling water as the value can easily be calculated using ambient temperature and pressure as well as tabulated data.
    The gas will only expand in this step and hence no air will get into the glass bulb.
    As soon as the helium has reached the desired temperature, the voltage is taken and the system is closed back up at opening 6. (see Fig.~\ref{fig_setup})
    From now on, the closed system can be used just like a thermometer when calculating the temperature at its corresponding voltage.
    
    To create a linear relation of pressure and temperature in the bulb, a second measurement is needed.
    The temperature at the freezing point of water is also known, so this is the second point used in our case.
    The Helium filled bulb is cooled in an ice bath and the voltage is taken.
    To calculate the absolute zero point of temperature, we have to pay respect to the expansion of the glass bulb with increasing temperature.
    Please refer to section~\ref{sec_results} for the calculation.

    As mentioned before, the closed system acts like a thermometer.
    Therefore, we can cool down the helium filled bulb using liquid nitrogen, read off the voltage and calculate the corresponding temperature of liquid nitrogen.
    In this step, the expansion of the glass bulb must be taken into account as well.
