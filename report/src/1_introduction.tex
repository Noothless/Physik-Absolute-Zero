\section{Introduction}
    Starting from the Boyle-Mariotte law $pV = \text{const}$, stated in 1661 by R. Townley, the question what happens if the pressure is changed while keeping the same volume is raised.
    G. Amontons delivered an answer to this question in 1703: pressure and temperature change are linked linearly.
    \begin{align}
        p(t) = \text{const} (t - t_0)
    \end{align}
    As pressures cannot be negative, it is now obvious that there must be a 'lowest' temperature, hereby referred to as the absolute zero point of temperature. 
    The findings also imply that pressure depends on temperature, amount of particles and volume, leading to the ideal gas law~\ref{eq_igl}.
    To determine the absolute zero point of temperature, we make use of the findings from Townley and Amontons:
    An apparatus of constant volume to which a pressure measurement device is connected can be heated up and cooled down, resulting in a change of pressure in the closed chamber.
    From these changes in pressure, a linear relation can be established which leads to the value of the lowest possible temperature $t_0$

    % In general this section should tell the reader why he or she should
    % be interested in your paper. Give some background to the
    % experiment, and describe the underlying principles. This is typically where you provide references to previous publications~\cite{Sato2003}.